\section{RESULTS}\label{sec:results}
The case studies presented in \cref{sec:case_studies}, using the method presented in \cref{sec:method}, were implemented on the Amigo robot.
Due to the system architecture already in place, the case studies could only be partly implemented.
It was thought that the existing SMACH finite-state machines of the robot system architecture, would provide an easy platform on which to integrate the supervisor.
However, the synchronous nature of these finite-state machines and their complexity, made it difficult to integrate them with the asynchronous callback structure of the supervisor.
An example of this is the grab skill SMACH finite-state machine in the Amigo repository.
The state machine navigates the robot in front of the specified object, updates the world model, opens and closes the grabber and more.
During the execution of this state machine the supervisor cannot call other events, but without all these steps the object will not be grabbed correctly.
These are just limitations on the robot side, however, and not limitations to the applicability and benefits of supervisory control synthesis to autonomous robot.
Furthermore, future research can solve these issues by, for example, making the callbacks multi-threaded.\\

Because of these limitations case study II could not be implemented on the robot, because the current architecture only allows planning of one arm at the same time.
Case study I was implemented for the most part.

More information on what is implemented and what is not, is provided in this paper's repository~\cite{jorrit_github}.
This implementation confirmed the following benefits the method in this paper discusses:\\

\begin{LaTeXdescription}
	\item [Concurrency] In case study I the robot moves one of the arms and navigates to a table at the same time.
	\item [Modularity and Re-usability] Both case studies combine a robot model, safety requirements, skill requirements and a task requirement from which a supervisor is synthesized. When the task requirement describing case study I is replaced with the task requirement describing case study II, synthesis provides a supervisor that executes this new task.\\
\end{LaTeXdescription}

And to a lesser extent:
\begin{LaTeXdescription}
	\item [Separation] Because the safety requirements were the same for both case studies, the effect of changing a requirement was not shown. Because of the modular design, however, this property should still hold.\\
\end{LaTeXdescription}



