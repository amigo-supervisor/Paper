\section{CONCLUDING REMARKS}\label{sec:conclusions}

In this paper, the limitations, the impact and benefits supervisory control synthesis can have in autonomous robots, are explored.
To this end, two case studies are described, using the Amigo robot at the Eindhoven University of Technology. 
A method is presented that can be used as a basis for implementation of supervisory control synthesis in autonomous robots.
This method defines on what abstraction levels supervisory control synthesis can be applied in autonomous robots and presents a framework for future research in the field.
The two case studies show that applying this method to autonomous robots indeed provides benefits in both system design and task execution.\\
\todo[disable]{limitations non-blocking}
Further research can increase the impact of supervisory control synthesis, by removing limitations mentioned in this paper.
On-line synthesis of the supervisor~\cite{online_partially_observed,online_near_optimal,online_sup_rep_obs_sub} increases its capability of reacting to changes in the dynamic environment of the robot.
Furthermore, using a non-monolithic supervisor is useful when the complexity and size of the system increases e.g.\@ modular~\cite{modular} or hierarchical~\cite{hierachical} synthesis.
Further research can also expand the implementaton of the case studies described in this paper, to provide further insight into the application of supervisory control synthesis to autonomous robots.\\

\todo[disable]{talk about making supervisor from modular requirements, coordinator and checking}