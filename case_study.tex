\section{CASE STUDIES}\label{sec:case_studies}
\Cref{sec:introduction} states the main contribution and purpose of this paper:
to investigate the limitations, benefits and application scope of supervisory control synthesis for autonomous robots.
To this end, two case studies were designed to showcase what supervisory control can contribute in the field of autonomous robotics.\\

As discussed in \cref{sec:introduction}, the benefits of supervisory control synthesis can be classified into two main categories: system design and task execution. 
Further on in \cref{sec:supervisory_control}, these categories were expanded on.
For systems design, aspects related to separation, modularity and re-usability were discussed.
The properties of the synthesized supervisor, non-blockingness and no violation of the requirements, are benefits in the task execution category.
The case studies presented here show the extent to which these benefits and principles can be applied to autonomous robots, using the method described in \cref{sec:method}.\\

The robot used for the case studies, is the Amigo robot, see~\cite{amigo_rop,amigo_ros,amigo_paper}. 
This is an autonomous service robot developed and maintained at the Eindhoven University of Technology in the Control Systems Technology Group. It competes in the RoboCup@Home league and has a functional system architecture and code, which can be found in the \texttt{tue\_robotics} repository~\cite{amigo_github}. 
This robot was chosen for the case studies, because the low-level control design was already in place and from there, it was possible to use supervisory control synthesis to control the robot with high-level commands. 
Also, the robot is controlled at task and skill level by a finite-state machine implementation in ROS using the SMACH library~\cite{smach}, which was thought to make interaction with the synthesized supervisor more straightforward.
The robot model, its requirements and the synthesized supervisor are created with the CIF 3 tool set~\cite{cif3}. CIF 3 is also developed and maintained at the Control Systems Technology Group.
CIF 3 can generate a C library~\cite{cif3_manual} implementation of the supervisor and this is leveraged to fuse the existing SMACH finite-state machines with the supervisor.
When the robot is being used without supervisor, a task is described by a SMACH finite state-machine.
In the case studies presented here, this task finite state-machine is replaced by a communication layer between the supervisor and the robot system architecture, called a wrapper. 
\Cref{fig:system_architecture} depicts the system architecture used for the case studies. \Cref{sec:method} provides a more detailed description of this architecture.\\

\begin{figure}[!ht]
\centering
% ->,>=stealth',
	\begin{tikzpicture}[node distance=2cm, auto,
                      semithick, text width=1.7cm, font=\footnotesize]
    % \node [circle, fill=black] (1){};
    % \node (2)[above of=1]{};
    % \node (3)[below of=1]{};

    \node[statesquaresup, align=left] (supervisor) {\textbf{Supervisor} \\
                                CIF 3};
    \node[statesquareimpl,left=0.4cm of supervisor, align=left] (smach) {\textbf{Wrapper} \\
                                              Communcation layer between the supervisor and existing system architecture};
    \node[statesquareimpl,above left=0.3cm and 0.3cm of smach, align=left] (skills) {\textbf{Robot skills} \\
                                               Robot abstraction layer of Amigo};
    \node[statesquareimpl,below left =0.3cm and 0.3cm of smach, align=left] (states) {\textbf{Robot states} \\
                                               SMACH skill specification of Amigo};
    \node[statesquarereq, align=left, above right=0.75cm and 0.5cm of supervisor] (req-safe) {\textbf{Requirements} \\
                                A CIF model of the safety requirements for Amigo};
    \node[statesquarereq, align=left, right=0.5cm of supervisor] (req-skill) {\textbf{Requirements} \\
                                A CIF model of the skill-specific requirements for Amigo};
    \node[statesquarereq, align=left, below right=0.75cm and 0.5cm of supervisor] (req-task) {\textbf{Requirements} \\
                                A CIF model of the task-specific requirements for Amigo};

    \draw [dashed] (-3.2,-9) -- (-3.2,9);

    % \draw [line, dashed] (2) -- (3);
    % \draw [line, ->] (1)  -- (smach);
    % \draw [line, ->] (1)  -- (supervisor);
    \draw [line, <->] (smach)  -- (supervisor);

    \draw [line] (skills)  -| (smach);
    \draw [line] (skills)  -- (states);
    \draw [line] (states)  -| (smach);
    \draw [line] (req-safe)  -| (supervisor);
    \draw [line] (req-skill)  -- (supervisor);
    \draw [line] (req-task)  -| (supervisor);

	\end{tikzpicture}
  \caption{System architecture}\label{fig:system_architecture}
\end{figure}

\subsection{CASE STUDY I:\@ PICK AND PLACE}\label{sec:cs1}
In this case study, Amigo navigates to a table with an object on it. 
It then grabs the object, navigates to a different table and puts the object down.
\Cref{fig:decision_flowchart_CSI} provides a, somewhat simplified, flowchart representation of the case study.
The supervisor here dictates when it is safe to execute the actions, move the arm, grab and navigate.\\

The system design benefits supervisory control synthesis provides in this case study, can be found in the way the code is structured. 
As can be seen in \cref{fig:system_architecture}, a clear distinction is made between safety, skill and task requirements. 
Apart from the model of the robot, there are 3 separate models of requirements, each with their own purpose:\\

\begin{LaTeXdescription}
  \item [Safety requirements] This set of requirements is always necessary for the robot to function correctly and needs to be included in the supervisory control synthesis for every task; e.g.\@ collision prevention. See \cref{fig:automaton_safe_arm_l} in Appendix~\ref{sec:model_req} and \cref{code:safeArmCIF} in Appendix~\ref{sec:code}. 
  \item [Skill requirements] This set of requirements is specific to a certain skill of the robot and needs to be included in the supervisor control synthesis of a task that uses this skill; e.g.\@ the grab skill. See \cref{code:grab_skill} in Appendix~\ref{sec:code}.
  \item [Task requirements] This set of requirements is specific to a certain task and defines the (order of) skills that need to be executed during a task. It needs to be designed for every task and placed on top of the safety and skill requirements; e.g.\@ the pick-up task. See \cref{fig:pick_task} in Appendix~\ref{sec:model_req} and \cref{code:pick_task} in Appendix~\ref{sec:code}.\\%,code:cs1-task-req
\end{LaTeXdescription}

% As mentioned before, besides task requirements, the system architecture proposed by this paper contains two more categories of requirements, skill and safety requirements.
% An example of a skill requirement used in this case study, is the grab skill, shown in \cref{code:grab_skill,fig:grab_skill}.
% An example of a safety requirement used in this case study, is collision avoidance, shown in \cref{fig:automaton_safe_arm_l}.\\

This separation makes the system very robust to changes. 
When, in the future, a new safety requirement needs to be implemented, the model of safety requirements is the only thing that needs to be changed. 
The newly synthesized supervisor will handle the rest, where without the use of supervisory control synthesis changes need to be made throughout the code.\\

One of the task execution benefits supervisory control synthesis provides in this case study, is that the individual skills are executed concurrently where possible.
% The code in SMACH is limited to just listing the needed skills in a concurrence state machines, these are a type of SMACH state machines that execute the states inside asynchronously~\cite{jorrit_github}.
As mentioned in \cref{sec:supervisory_control}, supervisory control synthesis can remove ownership of certain problems from the software developer to the synthesized supervisor.
In this case study, this means that the software developer only has to define the order of the skills needed for the case study in a CIF task requirement automaton.%, shown in \cref{fig:pick_task} in \cref{sec:model_req}
This requirement automaton describes the case study skill order as shown in \cref{fig:decision_flowchart_CSI}.
The supervisor then decides, based on these task requirements, the skill order.
It handles all skill scheduling and allows skills to execute concurrently where the requirements allow it.\\

Furthermore, this case study shows how the supervisor can handle unexpected situations. When a collision is found on the path, the arm stops moving and re-plans the path to the object. In \cref{sec:method}, a more in-depth explanation is given of the way the supervisor handles this.\\

\begin{figure}[!ht]
  \centering
  \begin{tikzpicture}[node distance = 3.5cm, auto, every node/.style={scale=0.8}]
    % Place nodes
    \node [block] (init) {Go to start position and wait};
    \node [cloud, left of=init] (user) {User};

    \node [circle, below of=init, fill=black, yshift=2cm] (comb1) {};
    \node [decision, below left of=comb1] (eval_nav1) {Supervisor can we navigate?};
    \node [decision, below of=comb1] (eval_manup1) {Supervisor can we move the arm?};
    \node [decision, below right of=comb1] (eval_grab1) {Supervisor can we grab?};

    \node [block, below of=eval_nav1] (navigate_to1) {Navigate to table 1};
    \node [block, below of=eval_manup1, yshift=1cm] (manup_to1) {Move arm to object};
    \node [block, below of=eval_grab1] (grab_to1) {Grab object};
    \node [circle, below of=manup_to1, fill=black, yshift=2.2cm] (comb2) {};

    \node [decision, below of=comb2, yshift=0.6cm] (succes1) {Is the object picked up?};
    \node [circle, below of=succes1, fill=black, yshift=1.8cm] (comb3) {};

    \node [decision, below left of=comb3] (eval_nav2) {Supervisor can we navigate?};
    \node [decision, below of=comb3] (eval_manup2) {Supervisor can we move the arm?};
    \node [decision, below right of=comb3] (eval_grab2) {Supervisor can we release?};

    \node [block, below of=eval_nav2] (navigate_to2) {Navigate to table 2};
    \node [block, below of=eval_manup2, yshift=1cm] (manup_to2) {Move arm to place point};
    \node [block, below of=eval_grab2] (grab_to2) {Release object};

    \node [circle, below of=manup_to2, fill=black, yshift=2.2cm] (comb4) {};

    \node [decision, below of= comb4, node distance=3cm] (succes2) {Is the object placed?};

    \node [block, below of=succes2, node distance=2.6cm] (stop) {stop};

    % Draw edges
    \path [line, dashed] (user) -- (init);
    \path [line] (init) -- (comb1);

    \path [line, anchor=south] (comb1)  -| (eval_nav1);
    \path [line, anchor=south] (comb1) -| (eval_manup1);
    \path [line, anchor=south] (comb1) -| (eval_grab1);

    \path [line, anchor=south] (eval_nav1)  -- node[decision answer] {yes} (navigate_to1);
    \path [line, anchor=south] (eval_manup1) -- node[decision answer] {yes} (manup_to1);
    \path [line, anchor=south] (eval_grab1) -- node[decision answer] {yes} (grab_to1);

    % \path [line, anchor=west, loop left] (eval_nav1) node[decision answer] {no} (eval_nav1);
    % \path [line, anchor=north, loop above] (eval_manup1) node[decision answer] {no} (eval_manup1);
    % \path [line, anchor=east, loop right] (eval_grab1) node[decision answer] {no} (eval_grab1);
    \path[->] (eval_nav1) edge [loop left, looseness=4] node {no} (eval_nav1);
    \path[->] (eval_manup1) edge [loop above, looseness=4] node {no} (eval_manup1);
    \path[->] (eval_grab1) edge [loop right, looseness=4] node {no} (eval_grab1);
    % \draw (eval_manup1) to [out=100,in=80,looseness=7] (eval_manup1);


    \path [line] (navigate_to1) |- (comb2);
    \path [line] (manup_to1) |- (comb2);
    \path [line] (grab_to1) |- (comb2);

    \path [line] (comb2) -- (succes1);

    \path[->] (succes1) edge [loop right, looseness=4] node {no} (succes1);
    \path [line, anchor=south] (succes1)  -- node[decision answer] {yes} (comb3);

    \path [line, anchor=south] (comb3) -| (eval_nav2);
    \path [line, anchor=south] (comb3) -| (eval_manup2);
    \path [line, anchor=south] (comb3) -| (eval_grab2);

    \path [line, anchor=south] (eval_nav2)  -- node[decision answer] {yes} (navigate_to2);
    \path [line, anchor=south] (eval_manup2) -- node[decision answer] {yes} (manup_to2);
    \path [line, anchor=south] (eval_grab2) -- node[decision answer] {yes} (grab_to2);

    \path[->] (eval_nav2) edge [loop left, looseness=4] node {no} (eval_nav2);
    \path[->] (eval_manup2) edge [loop above, looseness=4] node {no} (eval_manup2);
    \path[->] (eval_grab2) edge [loop right, looseness=4] node {no} (eval_grab2);

    \path [line] (navigate_to2) |- (comb4);
    \path [line] (manup_to2) |- (comb4);
    \path [line] (grab_to2) |- (comb4);

    \path [line] (comb4) -- (succes2);
    
    \path[->] (succes2) edge [loop right, looseness=4] node {no} (succes2);
    \path [line, anchor=south] (succes2)  -- node[decision answer] {yes} (stop);

  \end{tikzpicture}
  \caption{Decision flowchart of Case Study I}\label{fig:decision_flowchart_CSI}
\end{figure}

\subsection{CASE STUDY II:\@ DYNAMIC GRAB}\label{sec:cs2}
% The fact that the models of requirements can be applied to different tasks proves the \textit{modularity} and \textit{re-usability} of the supervisor control system.
This second case study is designed to show the \textit{modularity} and \textit{re-usability} of the models used for supervisory control synthesis.
Applying most of the same models, a new task with the same properties (i.e.\@ no collision) is designed with minimal effort.\\

In this case study, the robot starts in front of a table with two objects placed on it. The robot then picks up both objects.
In SMACH it is only specified that the robot has to pick up both objects, the grab-skill-requirement guarantees that a supervisor is synthesized that decides what arm is used to grab which object.
\cref{fig:decision_flowchart_CSII} shows the flowchart of the case study.\\

The safety requirements still apply, so collisions are still avoided (\cref{fig:automaton_safe_arm_l}), but the task requirements of \cref{sec:cs1} are removed. 

\begin{figure}[!ht]
  \centering
  \begin{tikzpicture}[node distance = 4cm, auto, every node/.style={scale=0.8}]
    % Place nodes
    \node [block] (init) {Go to table and wait};
    \node [cloud, left of=init] (user) {User};

    \node [decision, below of=init, yshift=1cm] (eval_arm_choice) {What arm is closer?};

    \node [decision, below left of=eval_arm_choice] (eval_arm_avail_l) {Supervisor is the arm free to use?};
    \node [decision, below of=eval_arm_avail_l] (eval_arm_safe_l) {Supervisor is the arm safe to move?};
    \node [decision, below right of=eval_arm_choice] (eval_arm_avail_r) {Supervisor is the arm free to use?};
    \node [decision, below of=eval_arm_avail_r] (eval_arm_safe_r) {Supervisor is the arm safe to move?};

    \node [block, below of=eval_arm_safe_l, yshift=1cm] (grab_to_l) {Grab object with left arm};
    \node [block, below of=eval_arm_safe_r, yshift=1cm] (grab_to_r) {Grab object with right arm};

    % Draw edges
    \path [line, dashed] (user) -- (init);
    \path [line] (init) -- (eval_arm_choice);

    \path [line, anchor=south] (eval_arm_choice)  -| node[decision answer] {left} (eval_arm_avail_l);
    \path [line, anchor=south] (eval_arm_choice) -| node[decision answer, right] {right} (eval_arm_avail_r);

    \path [line, anchor=south] (eval_arm_avail_l)  -- node[decision answer] {yes} (eval_arm_safe_l);
    \path[->] (eval_arm_avail_l) edge [loop left, looseness=4] node {no} (eval_arm_avail_l);
    \path [line, anchor=south] (eval_arm_avail_r) -- node[decision answer] {yes} (eval_arm_safe_r);
    \path[->] (eval_arm_avail_r) edge [loop right, looseness=4] node {no} (eval_arm_avail_r);

    \path [line, anchor=south] (eval_arm_safe_l)  -- node[decision answer] {yes} (grab_to_l);
    \path[->] (eval_arm_safe_l) edge [loop left, looseness=4] node {no} (eval_arm_safe_l);
    \path [line, anchor=south] (eval_arm_safe_r) -- node[decision answer] {yes} (grab_to_r);
    \path[->] (eval_arm_safe_r) edge [loop right, looseness=4] node {no} (eval_arm_safe_r);


  \end{tikzpicture}
  \caption{Decision flowchart of Case Study II}\label{fig:decision_flowchart_CSII}
\end{figure}
